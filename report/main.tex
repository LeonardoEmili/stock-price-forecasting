\documentclass{article}

\usepackage{booktabs}
\usepackage{multirow}
\usepackage{amsmath}
\usepackage{hyperref}
\usepackage{overpic}
\usepackage{amssymb}

\usepackage[accepted]{dlai2021}

%%% STUDENTS: FILL IN WITH YOUR OWN INFORMATION
\dlaititlerunning{Stock Price Forecasting using LSTM}

\begin{document}

\twocolumn[
%%% STUDENTS: FILL IN WITH YOUR OWN INFORMATION
\dlaititle{Stock Price Forecasting using LSTM}

\begin{center}\today\end{center}

\begin{dlaiauthorlist}
%%% STUDENTS: FILL IN WITH YOUR OWN INFORMATION
\dlaiauthor{Leonardo Emili}{}
\dlaiauthor{Alessio Luciani}{}
\end{dlaiauthorlist}

%%% STUDENTS: FILL IN WITH YOUR OWN INFORMATION
\dlaicorrespondingauthor{Leonardo Emili}{emili.1802989@studenti.uniroma1.it}
\dlaicorrespondingauthor{Alessio Luciani}{luciani.1797637@studenti.uniroma1.it}

\vskip 0.3in
]

\printAffiliationsAndNotice{}

%\begin{abstract}
%
%This is a \LaTeX template for writing your project report, to be submitted as part of the final exam. The template can not be modified (you can not change margins, spaces, etc.), and using this template is mandatory. Please read the main text for further details.
%
%\end{abstract}

% ------------------------------------------------------------------------------
\section{Introdution}

Stock price prediction is an active field of study where the objective goal is to predict
the trend of the stock market, typically based on its historical evidence. Despite the
theory of \emph{efficient markets} claims that there cannot be any pattern in the trend
of financial assets due to the overall knowledge of the mass, some affirm that periods
of human behavioral irrationality lead to strong operation correlation reflected in the
markets \cite{adaptive_markets}. Therefore, the task of predicting stock prices based
on past data seems not to be completely inapproachable. Over the years,
many methods have been proposed to tackle the task: ranging from the Naive Forecast approach, which trivially forecasts the stock value to be the last observed one, up to the most recent ones that heavily rely on machine learning techniques. In this context, we will delve into the topic of stock price prediction using fundamental data to assess whether a stock is attractive to investors. 
This technique, as opposed to technical analysis, considers prices, as well as financial reports, to model the problem. Here, the underlying assumption is that one cannot tell whether it's worth buying a stock only by looking at its current price and volumes. Indeed, quarterly released financial reports about companies and external information (e.g. the common sentiment on a stock ticker) may be crucial to assess the quality of a given stock. In this project, we experimented with the effectiveness of deep learning techniques applied to the stock price prediction task. The key idea is the use of recurrent neural networks that are able to exploit temporal dependencies of events, hence conditioning the presence of an event at timestep \emph{t} on the previous events at \emph{t-1, t-2, \ldots}. 

% ------------------------------------------------------------------------------
\section{Related work}

Some approaches have been explored in this direction, such as in \cite{32066186} where the authors apply LSTM, Stacked-LSTM, and Attention-Based LSTM into the prediction of stock prices. The authors also propose an evaluating framework to assess the quality of their models based on the return of the trading strategy. In another work \cite{mehtab2020stock}, the authors predict the open value of NIFTY 50 using different machine learning and deep learning models. They also demonstrate that using one-week prior data as input leads to good results.

% ------------------------------------------------------------------------------
\section{Dataset}

Data wrangling operations have been crucial for this task. In fact, we start considering the S\&P 500 stock data, which is a collection of daily stock prices for all companies from the S\&P 500 index. This dataset provides us with the following features: open price, highest price, lowest price, close price, volume, stock ticker, and date. We also consider an auxiliary dataset that provides us with fundamental data from Yahoo Finance. The dataset contains the following columns: Forward P/E, DE Ratio, Earnings Growth, Enterprise Value/EBITDA, EBITDA, Current Ratio, Cash Flow, Trailing P/E, Beta, PEG Ratio, Gross Profit, Total Debt, Price, Return on Equity, Return on Assets, Price/Book, Revenue Growth, Operating Margin, Enterprise Value/Revenue, Revenue, Total Cash, Enterprise Value, Total Cash Per Share, Profit Margin, Price/Sales, Book Value Per Share, Diluted EPS, Market Cap, Revenue Per Share, Net Income Avl to Common, Ticker, Date. Based on the date column, we can align the two datasets such that for each event in the S\&P 500 dataset, we have the latest available financial report. From now on, we will refer to the dataset obtained from the alignment process simply as the dataset.

\subsection{Feature engineering}

We fill forward values whenever possible, we otherwise replace missing values with constant values (i.e. zero paddings). The intuition here is that if there are missing values at the beginning of the history of a stock, it means that the feature is not available, otherwise referring to the latest value as the most up-to-date one. As an example, consider holidays when markets are closed, and the price does not change since no orders are placed. We also fill the dataset with missing working days using the same forward-fill strategy.
Furthermore, we perform significant feature engineering steps adding technical indicators
such as SMA and RSI. The SMA is computed by averaging the prices of a given number of multiple
contiguous time steps. The RSI was, instead, computed using the relative strength formula
\cite{rsi}.

$$
U = max(0, closeNow - closePrevious)
$$
$$
D = max(0, closePrevious - closeNow)
$$
$$
RS = SMMA(U,n)/SMMA(D,n)
$$
$$
RSI = 100 - 100/(1+RS)
$$

It takes into account the differences in contiguous prices, with respect to the
exponentially smoothed moving average. Therefore, it makes it easy to spot overbought
and oversold events. In fact, using this technical indicator, we could extract the
overbought and oversold binary features too. According to the RSI definition,
a value that goes higher than 70 can be interpreted as a situation of overbought.
Similarly, a value that goes below 30 is seen as a situation of oversold. An articulated
model could of course extract this kind of information directly from the price, but manually
adding these knownly meaningful indicators definitely simplifies the work of the network.
Another step of the feature engineering process involved scaling the features to put them
on similar scales. This step was very important considering that we are using gradient
descent to optimize the models and having features with completely different scales would
have made the convergence task much harder. Therefore, we globally scaled prices,
volumes, technical indicator values, fundamentals, etc. At this point, all the features
were centered on the same scale.
Before training the model, we split the dataset into three subsets: \emph{train set},
\emph{dev set} and \emph{test set}. These are respectively meant for training the model,
tuning its hyperparameters and testing its performance. Since the core idea of the project
is forecasting future events by looking at past data, we cannot leak future information in
the train set. Therefore, we split the dataset by years. Having temporal data ranging from
2004 to 2013, we dedicate the years 2004 - 2011 to the train set, 2012 to the dev set, and
2013 to the test set. This way, the procedure is similar to backtesting the model strategy
on present events.
In order to predict the target adjusted close price, we consider the dataset according to the sliding window approach, also known as the lag method. In this way, we decompose the original dataset with overlapping windows and condition the target value only on its lag. During the training phase, we treat both the step size (i.e. the number of days before a new window) and the window size (i.e. the size of the lag) as hyperparameters and tune them accordingly.

% ------------------------------------------------------------------------------
\section{Models}

In this project, we apply the sliding window approach with deep recurrent neural networks. As our first models, we start with naive recurrent networks, respectively LSTM and GRU models. As opposed to vanilla RNN, LSTM and GRU networks partially solve the problem of vanishing gradient by employing a gating mechanism to regulate the amount of information to carry from previous time steps \cite{vanishing}. Our input has the following shape: (\emph{batch\_size}, \emph{window\_size}, \emph{feature\_size}). Moreover, we assume that there exists a local pattern that we can leverage to predict future prices.
In fact, in a third model, we make use of CNN layers to extract such information across
time steps. It consists of a convolutional layer followed by an LSTM layer, then connected
to two dense layers. The convolutional layer is meant to extract high level patterns that
form across time steps inside a window. This information is then also passed through the
recurrent unit in order to get enriched by means of temporal sequencial correlation.
As a fourth model, we employ the attention mechanism over our input
data and then feed it to an LSTM layer. In particular, we employ a multi-head
self-attention mechanism \emph{to jointly attend to information from different
representation subspaces at different positions} \cite{46201}. Similarly to the previous
model, here the idea is to process the raw input sequence by enriching it via a preliminary
layer, and then extracting its sequential information via the LSTM layer. However, in this
case the attention layer operates differently compared to the CNN one. In fact, it works
by putting the focus on specific parts of the input sequence. For this task, it can be useful
since it can understand that specific subsequences of financial data are more meaningful than
others and give those more importance.    

% ------------------------------------------------------------------------------
\section{Hyperparameters}

In this section, we present a subset of the hyperparameters used. However, it is not intended to be an exhaustive list of all the hyperparameter. The list of runs is logged using wandb \cite{wandb} and can be reached at \href{https://wandb.ai/leonardoemili/spf?workspace=user-leonardoemili}{link}.

\begin{table}[h!]
    \caption{List of hyperparameters.}
    \label{tab:hparams}
    \begin{center}
    \begin{small}
    \begin{tabular}{p{2cm}p{1cm}}
    \toprule \multirow{2}{2cm}{Hyperparameter} & \multirow{2}{0.1\linewidth}{Value}\\
    \\
    \midrule
    Window size & 20 \\
    Step size & 1 \\
    Optimizer & SGD \\
    Batch size & 1024 \\
    \bottomrule
    \end{tabular}
    \end{small}
    \end{center}
    \vspace{-0.5cm}
\end{table}

% ------------------------------------------------------------------------------
\section{Experimental results}

As an evaluation framework, we consider multiple metrics in order to assess the quality
of a particular model. Since we are dealing with a regression task it comes naturally
to adopt the Mean Squared Error (MSE) measure as a proxy to indicate how well a model
performs. We also consider the R-squared statistical measure to check how similar
the predicted and the observed data are similar. However, since we are aware
that R-squared increases when adding more independent variables, we use the
adjusted R-squared. Its formulation is similar to the one of R-squared and only
differs from the degrees of freedom. Finally, we measure the real performances
of our system according to the return of a trading strategy. The system is
designed to buy a fixed amount of stocks whenever the predicted price of
the following day is higher than the current one. Then, it closes the position after one
day and registers the gain or the loss. This way, the profit expectancies can be compared
among all the models. Another metric that we took into consideration was the operation's
accuracy. That is an unconventional metric that we decided to adopt to understand the
fraction of times that the model predicted the correct trend. In other words, when the model
predicted a price gain and the real outcome was actually a gain. The reason why we
implemented these additional evaluation steps is because regression metrics were not enough
to assess the goodness of the model in performing this task. In fact, a model could
simply replicate the price of the previous day by applying the identity function and obtain
decent MSE. This is because the relative change in price between two subsequent days is
very small on average. However, such a model would make very careless predictions that
would be of no help to a potential investor. \\
In table \ref{tab:results} are shown the quantitative results obtained with the models.
Considering the regression metrics, we can see that our novel architectures (i.e. self-attention
and convolutional) outpeformed the standard recurrent units. In fact, the attention-based
LSTM reached the very best results in those terms. On the other hand, the differences in
profit were negligible among the various methods.
\underline{\textbf{TODO:}} explain what are our best performing models and the reasons why they perform so well.

\begin{table}[h!]
    \caption{Performance comparison.}
    \label{tab:results}
    \begin{center}
    \begin{small}
    %\begin{tabular}{p{0.26\linewidth} | ccccc}
    \begin{tabular}{p{1.7cm}p{0.55cm}p{0.55cm}p{0.9cm}p{1.7cm}p{0.8cm}}
    \toprule
    & \multirow{2}{0.1\linewidth}{MSE}& \multirow{2}{0.1\linewidth}{R2}& \multirow{2}{0.1\linewidth}{Adjusted R2}& \multirow{2}{\linewidth}{Operation accuracy (\%)}& \multirow{2}{0.1\linewidth}{Return (\%)}\\
    \\
    \midrule
    Naive LSTM      & 0.045 & 0.985 & 0.955 & \textbf{85.79} &  419.12  \\
    Naive GRU & 0.062 & 0.988 & 0.970 & 85.11 &  \textbf{422.87}  \\
    CNN-LSTM & 0.033 & 0.988 & 0.968 & 83.52 &  390.87  \\
    Attention-based LSTM & \textbf{0.022} & \textbf{0.991} & \textbf{0.977} & 0.8102 &  420.26  \\
    \bottomrule
    \end{tabular}
    \end{small}
    \end{center}
    \vspace{-0.5cm}
\end{table}

% ------------------------------------------------------------------------------
\section{Conclusions}

We have seen a stock price prediction deep learning approach that brought several
recurrence-based architectures to the table. At the beginning of this work we questioned
ourself about the feasibility of such predictions that go against the efficient markets'
theory. We demonstrated that a model is actually capable of extracting some patterns and
use them to make profitable decisions. Thus, markets may have been not so efficient during
the periods that are present in our dataset, especially in the 2013 test set. We would
argue that the predictions made by these models could vary very much in profitability
in different periods in time, affected by different events and circumstances. Furthermore,
a potential widespread use of similar models by large istitutions and individual investors
can eventually annihilate their effectiveness, since the information extracted by them would
be reflected in the prices. So, in order to keep the same levels of profit, possible
alternatives could be more sophisticated architectures or more prior assumptions.  

\underline{\textbf{TODO:}} Add conclusions here ..

% Cool image with caption example
%\begin{figure}[t]
%    \centering
%    \begin{overpic}[width=0.99\linewidth]{./torus.png}
%    \put(-1, 21){\color{blue}\footnotesize $\mathcal{M}_2$ }
%    \put(13, 12){\color{red}\footnotesize $\mathcal{M}_1$ }
%    \put(93, 30){\footnotesize $\mathcal{Z}$ }
%    \put(79, 26){\scriptsize $z_2$ }
%    \put(88, 26){\scriptsize $z_1$ }
%    \end{overpic}
%    \caption{In the figure caption, you can write what you want including formulas, e.g. $\mathcal{X} \subset \mathbb{R}^3$. Notice that in this figure, we added mathematical symbols on top of the image by using the overpic command.}
%    \label{fig:torus}
%\end{figure}

\bibliography{references.bib}
\bibliographystyle{dlai2021}

\end{document}

